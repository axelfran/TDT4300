\chapter{Exploring Data}
	
	\clearpage
	{\bf Summary Statistics:} 

	Are quantiles, such as the mean
	and standard deviation, that capture various characteristics of a 
	potentially large set of values with a single number or a small set
	of numbers. 

	{\bf Frequencies and the Mode:} 

	The Mode of a categorical attribute is the value that has the highest frequency.
	The frequency of a value is defined by:	
	\begin{equation}
		frequency(v_{i}) = \frac{number of objects with attribute value v_{i}}{total objects}
	\end{equation}  

	{\bf Percentiles:}

	{\bf Mean and Median:}

	For continupus data, two of the most widley used summary statistics are the mean
	and the median, which are measures of the location of a set of values.

	\begin{equation}
		mean(x) = \overline{x} = \frac{1}{m} \sum_{i=1}^{m} x_i
	\end{equation}

	{\bf Range and Variance}

	\begin{equation}
		range(x) = max(x) - min(x) = x_{m} - x_{1}
	\end{equation}	

	\begin{equation}
		variance(x) = s_{x}^{2} = \frac{1}{m - 1}\sum_{i=1}^{m} (x_{i} - \overline{x})^{2}
	\end{equation}	

	{\bf ADD, MAD and IQR}

	sometimes do we need a more robust estimate of the spread of a set because of outliners.
	Here are some alternatives:

		\begin{equation}
			ADD(x) = Absolute Average Deviation = \frac{1}{m} \sum_{i=1}^{m} |x_{i} - \overline{x}|
		\end{equation}

		\begin{equation}
			MAD(x) = Median Absolute Deviation = median (\{|x_{1} -\overline{x}|, ..... , 
			|x_{m} -\overline{x}|\})
		\end{equation}

		\begin{equation}
			interquartile range(x) = x_{75\%} - x_{25\%}
		\end{equation}

	{\bf Multivariate Symmary Statistics}

		\begin{equation}
			covariance(x_{i}, x_{j}) = \frac{1}{1 - m} \sum_{k=1}^{m} (x_{ki} - \overline{x_{i}})(x_{kj} -
			 \overline{x_{j}}) 
		\end{equation}

		\begin{equation}
			correlation(x_{i}, x_{j}) = \frac{covariance(x_{i}, x_{j})}{s_{i}s_{j}}
		\end{equation}

	\clearpage
	\section{Visualization}

	The use of visualization techniques in data mining is refferred to as 
	{\bf visual data mining}.

	The overriding motivation for using visualization is that people can quickly absorb
	large amounts of visual information and find patterns in it. 

	\subsection{General Concepts}
		{\bf Representation: Mapping Data to Graphical Elements}

		{\bf Arrangements}

		{\bf Selection}

	\subsection{Techniques}

		{\bf Visualizing Small Numbers of Attributes}

		{\bf Stem and Leaf Plots}

		{\bf Histograms}

		{\bf Box Plots}

		{\bf Pie Chart}

		{\bf Scatter Plots}

		{\bf Contour Plots}

		{\bf Surface Plots}

		{\bf Vector Field Plots}

		{\bf Animation}

	\subsection{Visualizing Higer-Dimensional Data}
		{\bf Metrices}

		{\bf Parallel Coordinates}

		{\bf Star Coordinates and Chernoff Faces}

	\subsection{Do's and Don'ts}

		{\bf ACCENT Principles:} 
			\begin{itemize}
				\item {\bf Apprehension:}
				\item {\bf Clarity:}
				\item {\bf Consistency:}
				\item{\bf Efficiency:}
				\item {\bf Necessity:}
				\item {\bf Truthfulness:}
			\end{itemize}

		{\bf Tufte's Guidelines}
			\begin{itemize}
				\item Graphical excellence is the well-designed presentation of
				interesting data a matter of substance, of statistics, and of design.
				\item GRaphical exellence consists of complex ideas communicated 
				with clarity, precision, and efficiency.
				\item Graphical excellence is that which gives to the viewer the greatest
				number of ideas in the shortest time with the least ink in the smallest space.
				\item Graphical excellence is nearly always multivariate
				\item And graphical excellence requires telling the truth about the data.
			\end{itemize}

	\section{OLAP and Multidimensional Data Analysis}

		Whats important are the insights can be gained by looking at data from
		a multidimentional viewpoint.
		The starting point is usually a tabular representation of the data, such 
		as a {\bf fact table}. 

		Two steps are necessary in order to represent data as a multidimentional 
		array:
			\begin{enumerate}
				\item identification of the dimensions
				\item identification of an attribute that is the focus of the analysis.
			\end{enumerate}

		The values of an attribute seve as indices into the array for the dimension
		corresponding to the attribute, and the number of attribute values in the size
		of that dimension. 

		The content of each cell represents the value of a {\bf taget quantity} that
		we are interested in analyzing.

		\subsection{Analyzing Multidimensional Data}

		The key motivation for taking a multidimensional viewpoiint of data is the 
		importance of aggregating data in various ways. 

		A multidimensional representation of the data, together with all possible totals
		(aggregates), is known as a {\bf data cube}.
		A data cube may have either more or fewer than three dimensions. A data cube
		is a generalization of what is known in statistical terminology as a 
		{\bf cross-tabulation}.

		{\bf Dimensionality Reduction and Pivoting:}
		\begin{itemize}
			\item {\bf Aggregation} can be viewed as a form of aggregation.
			\item{\bf Pivoting} refers to aggregating over all dimensions except two. The
			result is a two-dimensional cross tabulation with the two specified dimensions
			as the only remaining dimensions. 
		\end{itemize}

		{\bf Slicing and Dicing}

			\begin{itemize}
				\item {\bf Slicing} is selecting a group of cells from the entire 
				multidimensional array by specifying a specific value for one or more
				dimensions.
				\item {\bf Dicing} involves seleccting a subset of cells by specifying a 
				range of attribute values. This is equivalent to defining a subarray
				from the complete array. In practice, both operations can also be accompanied
				by aggregation over some dimensions. 
			\end{itemize}			


		{\bf Roll-Up and Drill-Down}

			Categories (attributes) can often be organized as a hierarchical tree or lattice.
			For instance, years consist of months and weeks, both of which consist of days.
			Locations can be divided into nations, which contains states, which in turn
			contains cities. Likewise of products can be further subdivided. For example,
			the product category, funiture, can be subdivided into the subcategories,
			chairs, tables, sofas, etc. This hierchical structure gives rise to  the 
			roll-up and drill-down operations. To illustrate, starting with the original
			sales data, which is a multidimensional array with entries for each date,
			we can aggregate {\bf (roll-up)} the sales across all the dates in a month. 
			Conversely, given a representation of the data where the times dimensions are
			broken into months, we might want to split the monthly sales totals 
			{\bf (drill-down)} into daily sales. 

			Thus, roll-up and drill-down operations are related to aggregation. Notice, 
			however, that they differ from the aggregation operations from last section
			because they aggregate cells within a dimension, not across the entire  dimension. 
			