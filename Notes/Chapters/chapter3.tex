\chapter{Exploring Data}
	
	\clearpage
	{\bf Summary Statistics:} 

	Are quantiles, such as the mean
	and standard deviation, that capture various characteristics of a 
	potentially large set of values with a single number or a small set
	of numbers. 

	{\bf Frequencies and the Mode:} 

	The Mode of a categorical attribute is the value that has the highest frequency.
	The frequency of a value is defined by:	
	\begin{equation}
		frequency(v_{i}) = \frac{number of objects with attribute value v_{i}}{total objects}
	\end{equation}  

	{\bf Percentiles:}

	{\bf Mean and Median:}

	For continupus data, two of the most widley used summary statistics are the mean
	and the median, which are measures of the location of a set of values.

	\begin{equation}
		mean(x) = \overline{x} = \frac{1}{m} \sum_{i=1}^{m} x_i
	\end{equation}

	{\bf Range and Variance}

	\begin{equation}
		range(x) = max(x) - min(x) = x_{m} - x_{1}
	\end{equation}	

	\begin{equation}
		variance(x) = s_{x}^{2} = \frac{1}{m - 1}\sum_{i=1}^{m} (x_{i} - \overline{x})^{2}
	\end{equation}	

	{\bf ADD, MAD and IQR}

	sometimes do we need a more robust estimate of the spread of a set because of outliners.
	Here are some alternatives:

		\begin{equation}
			ADD(x) = Absolute Average Deviation = \frac{1}{m} \sum_{i=1}^{m} |x_{i} - \overline{x}|
		\end{equation}

		\begin{equation}
			MAD(x) = Median Absolute Deviation = median (\{|x_{1} -\overline{x}|, ..... , 
			|x_{m} -\overline{x}|\})
		\end{equation}

		\begin{equation}
			interquartile range(x) = x_{75\%} - x_{25\%}
		\end{equation}

	{\bf Multivariate Symmary Statistics}

		\begin{equation}
			covariance(x_{i}, x_{j}) = \frac{1}{1 - m} \sum_{k=1}^{m} (x_{ki} - \overline{x_{i}})(x_{kj} -
			 \overline{x_{j}}) 
		\end{equation}

		\begin{equation}
			correlation(x_{i}, x_{j}) = \frac{covariance(x_{i}, x_{j})}{s_{i}s_{j}}
		\end{equation}

	\clearpage
	\section{Visualization}

	The use of visualization techniques in data mining is refferred to as 
	{\bf visual data mining}.

	The overriding motivation for using visualization is that people can quickly absorb
	large amounts of visual information and find patterns in it. 

	\subsection{General Concepts}
		{\bf Representation: Mapping Data to Graphical Elements}

		{\bf Arrangements}

		{\bf Selection}

	\subsection{Techniques}

		{\bf Visualizing Small Numbers of Attributes}

		{\bf Stem and Leaf Plots}

		{\bf Histograms}

		{\bf Box Plots}

		{\bf Pie Chart}

		{\bf Scatter Plots}

		{\bf Contour Plots}

		{\bf Surface Plots}

		{\bf Vector Field Plots}

		{\bf Animation}

	\subsection{Visualizing Higer-Dimensional Data}
		{\bf Metrices}

		{\bf Parallel Coordinates}

		{\bf Star Coordinates and Chernoff Faces}

	\subsection{Do's and Don'ts}

		{\bf ACCENT Principles:} 
			\begin{itemize}
				\item {\bf Apprehension:}
				\item {\bf Clarity:}
				\item {\bf Consistency:}
				\item{\bf Efficiency:}
				\item {\bf Necessity:}
				\item {\bf Truthfulness:}
			\end{itemize}

		{\bf Tufte's Guidelines}
			\begin{itemize}
				\item Graphical excellence is the well-designed presentation of
				interesting data a matter of substance, of statistics, and of design.
				\item GRaphical exellence consists of complex ideas communicated 
				with clarity, precision, and efficiency.
				\item Graphical excellence is that which gives to the viewer the greatest
				number of ideas in the shortest time with the least ink in the smallest space.
				\item Graphical excellence is nearly always multivariate
				\item And graphical excellence requires telling the truth about the data.
			\end{itemize}

	\section{OLAP and Multidimensional Data Analysis}