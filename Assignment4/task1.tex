\section{Repitition: Apriori Algorithm}

	{\bf Given the shopping basketin table 1, use the apriori algorithm to generate
	all possible association rules (for minimum support 0.5 and minimum confidence 0.8).
	Provide step-by-step notes on how you reach your result.}

	\begin{table}[H]
		\centering
		\begin{tabular}{ c | c }
			\hline
			{\bf ID} & {\bf Transaction} \\ \hline
			1 & A, B, C \\
			2 & A, C \\
			4 & A, D \\
			5 & B, E, F \\ \hline
		\end{tabular}
		\caption{Shopping basket}
	\end{table}
	
	\noindent\makebox[\linewidth]{\rule{\textwidth}{1pt}} 


	\section*{Support - Generating Frequent Itemset}

	First we begin with calculating the support for alle the single items from
	table1. We want to count how many transactions thar contains each item. 
	
	\begin{wraptable}{r}{5.3cm}
		\begin{tabular}{ c | c | c }
			\hline 
			{\bf Item} & {\bf Count} & {\bf Support} \\ \hline
			A & 3 & 3/4 = 0.75 \\
			B & 2 & 2/4 = 0.50 \\
			C & 2 & 2/4 = 0.50 \\			
			D & 1 & {\color{red}1/4 = 0.25} \\
			E & 1 & {\color{red}1/4 = 0.25} \\
			F & 1 & {\color{red}1/4 = 0.25} \\ \hline
		\end{tabular}
	\end{wraptable}
	To be added to the frequent itemset, the item needs a minimum support of 0.50.
	The items that do not have a minimum support of 0.5 is colored in red. 
	The support is calculated by dividing the count by the number of transactions.
	According to the calculations, we can add A, B and C to our frequent itemset
	and prune all the items with support lower than the minimum support.
	The next step in the calculation is to combine the items in the frequent itemlist
	with each other. 
	\begin{wraptable}{l}{5.3cm}
		\begin{tabular}{ c | c | c }
			\hline 
			{\bf Item} & {\bf Count} & {\bf Support} \\ \hline
			A B & 1 & {\color{red}1/4 = 0.25} \\
			A C & 2 & 2/4 = 0.50 \\
			B C & 1 & {\color{red}1/4 = 0.25} \\ \hline			
			
		\end{tabular}
	\end{wraptable}

	The  result is a list with 2-itemlists. We now repeat the process
	by counting in how many transactions that contains each itemset. The minimum support 
	are still 0.50. As we can see, the only set that got a minimum support was the
	itemset AC. We add AC to our frequent itemset and prune the itemsets with support
	lower than the minimum support. 
	

	\begin{wraptable}{r}{5.3cm}
		\begin{tabular}{ c | c | c }
			\hline 
			{\bf Item} & {\bf Count} & {\bf Support} \\ \hline
			A B C & 1 & {\color{red} 1/4 = 0.25} \\ \hline
		\end{tabular}
	\end{wraptable}

	We will continue this process until we only get the empty set. We still got elements
	left. If we combine out frequent 1-itemsets and frequent 2-itemsets, we can get
	3-itemsets. Combining AC with A, B and C only gives 1 distinct set ABC. When we calculate
	the support, we get a support lower than the minimum support, and the itemset is pruned. 
	There is no itemsets left, we got the empty itemset, and we will stop here. We finally
	calculated our frequent itemset containing the itemsets: A, B, C, and AC.

	{\bf Frequent itemset} = \{A, B, C, AC\}
	
	\subsection*{Confidence - Rule Generation}

	Since we got our frequent itemsets, we can generate rules by calculating the confidence
	combining our itemsets into rules. Rules with a confidence greater than 0.8 is added to
	our set with rules. Permutations of the frequent itemlist have the notation X $\rightarrow$
	Y. The table below is the calculation of the confidence. We have listed all ther permutations,
	the number of transactions that contains X and Y, number of transactions that contains X, and
	the calculataed confidence. 

		\begin{table}[H]'
		\centering
			\begin{tabular}{ c | c | c | c }
				\hline
				Permutations & X and Y & X & Confidence \\ \hline
				A $\rightarrow$ B & 1 & 3 & {\color{red} 1/3 = 0.33} \\ 
				A $\rightarrow$ C & 2 & 3 & {\color{red} 2/3 = 0.67} \\ 
				A $\rightarrow$ AC & 2 & 3 & {\color{red} 2/3 = 0.67} \\ 
				B $\rightarrow$ A & 1 & 2 & {\color{red} 1/2 = 0.5} \\ 
				B $\rightarrow$ C & 1 & 2 & {\color{red} 1/2 = 0.5} \\ 
				B $\rightarrow$ AC & 1 & 2 & {\color{red} 1/2 = 0.5} \\ 
				C $\rightarrow$ A & 2 & 2 & 2/2 = 1.0 \\ 
				C $\rightarrow$ B & 1 & 2 & {\color{red} 1/2 = 0.5} \\
				C $\rightarrow$ AC & 2 & 2 & 2/2 = 1.0 \\
				AC $\rightarrow$ A & 2 & 2 & 2/2 = 1.0 \\
				AC $\rightarrow$ B & 1 & 2 & {\color{red} 1/2 = 0.5} \\
				AC $\rightarrow$ C & 2 & 2 & 2/2 = 1.0 \\ \hline
			\end{tabular}
		\end{table}

	From the calculations we can see that there is 4 generated rule that is over the minimum
	confidence threshold. The rules are C $\rightarrow$ A, C $\rightarrow$ AC, AC $\rightarrow$ A and
	AC $\rightarrow$ C.


	

