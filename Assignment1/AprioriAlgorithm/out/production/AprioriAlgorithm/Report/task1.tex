\section{Implementation of Frequent Itemset Methods}
	
	In this exercise we implemented 3 different methods for generating the frequent 
	itemsets. We will in describe the implementation of each in this section.

	\subsection{Brute-Force}
		This method is quite simple; it genereates all possible combination and 
		prune candidates that have support less than the threshold (minsupport), which
		is 0.4 in this implementation. 

	\subsection{FKminus1F1}

	\subsection{FKminus1FKminus1}


\section{Scaleability of Implemented Methods}
	
	\subsection{Small Data Set}
	\begin{table}[H]
		\begin{tabular}{| p{3.5cm} | p{2.5cm} | p{2.5cm} | p{2.5cm} | p{2.5cm} | }
			\hline
			Method & Level 1 & Level 2 & Level 3 & Level 4 \\ \hline
			Brute-Force & Generated 6 kept 4 & Generated 15, kept 4 & Generated 20, kept 4 & Generated 20, kept 1 \\ \hline
			FKminus1F1 & Generated 6, kept 4 & Generated 6, kept 4 & Generated 4, kept 1 & - \\ \hline
			FKminus1FKminus1 & Generated 6, kept 4 & Generated 6, kept 4 & Generated 1, kept 1 & - \\ \hline
		\end{tabular}
	\end{table}

	\subsection{Big Data Set}
	\begin{table}[H]
		\begin{tabular}{| p{3.5cm} | p{2.5cm} | p{2.5cm} | p{2.5cm} | p{2.5cm} | }
			\hline
			Method & Level 1 & Level 2 & Level 3 & Level 4 \\ \hline
			Brute-Force & Generated 6kept 4 & Generated 15, kept 4 & Generated 20, kept 4 & Generated 20, kept 1 \\ \hline
			FKminus1F1 & Generated 122, kept 17 & Generated 136, kept 15 & - & - \\ \hline
			FKminus1FKminus1 & Generated 122, kept 17 & Generated 136, kept 15 & - & - \\ \hline
		\end{tabular}
	\end{table}


